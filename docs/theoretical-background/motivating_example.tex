\usepackage{subcaption}
\usepackage{xcolor, colortbl}
\usepackage{amsfonts}
\usepackage{amsmath,thm-restate}
\usepackage{amssymb}
\usepackage{temporal}
\usepackage{cite}
\usepackage{graphicx}
\usepackage[inline]{enumitem}
\usepackage{wrapfig}
\usepackage{environ}
\usepackage{multirow}
\usepackage{xspace}
\usepackage{color}
\usepackage{thmtools}
\usepackage{makecell}
\usepackage{pgf}
\usepackage{caption}
\usepackage{mdframed}
\definecolor{mygreen}{RGB}{0,150,0}
\usepackage{pifont} % symboles
\newcommand{\xmark}{\leavevmode{\color{red}\ding{55}}}%
\newcommand{\cmark}{\leavevmode{\color{green!75!black}\ding{51}}}%


\newcommand{\mailto}[1]{\href{mailto:#1}{\nolinkurl{#1}}}

% general logic notation
\newcommand{\modelsg}[1]{\models_{#1}}
\newcommand{\mT}{\mathcal{T}}
\newcommand{\mTz}{\mathcal{T}^{z}}
\newcommand{\mR}{\mathcal{R}}
\newcommand{\mI}{\mathcal{I}}
\newcommand{\mC}{\mathcal{C}}
\newcommand{\mD}{\mathcal{D}}
\newcommand{\mF}{\mathcal{F}}
\newcommand{\mX}{\mathcal{X}}
\newcommand{\mM}{\mathcal{M}}
\newcommand{\mN}{\mathcal{N}}
\newcommand{\mA}{\mathcal{A}}
\newcommand{\mP}{\mathcal{P}}
\newcommand{\mB}{\mathcal{B}}
\newcommand{\mU}{\mathcal{U}}
\newcommand{\mS}{\mathcal{S}}
\newcommand{\mE}{\mathcal{E}}
\newcommand{\mQ}{\mathcal{Q}}
\newcommand{\mAR}{\mathcal{AR}}
\newcommand{\mTH}{\mathcal{TH}}
\newcommand{\bbN}{\mathbb{N}}
\newcommand{\bbB}{\mathbb{B}}
\newcommand{\bbR}{\mathbb{R}}


% general logic notation
\newcommand{\vec}[1]{\mathbf{#1}}

% (non-abstract) threshold automata
\newcommand{\resets}{\tau}
\newcommand{\Shared}{\Gamma}
\newcommand{\vP}{\vec{P}}
\newcommand{\vp}{\vec{p}}
\newcommand{\from}{\textit{from}}
\newcommand{\to}{\textit{to}}
\newcommand{\vup}{\vec{uv}}
\newcommand{\Rules}{\mR}
\newcommand{\Paths}{\mP(TA)}


% (non-abstract) counter systems
\newcommand{\cs}{\mathsf{CS}}
\newcommand{\csa}{\mathsf{CS(A)}}
\newcommand{\vg}{\vec{g}}
\newcommand{\vk}{\vec{k}}
\newcommand{\Confs}{\Sigma}
\newcommand{\conf}{\sigma}
\newcommand{\Trans}{\mT}

% abstract threshold automata
\newcommand{\absTA}{\overline{\text{TA}}}
\newcommand{\absA}{\overline{A}}
\newcommand{\absShared}{\overline{\Shared}}
\newcommand{\absRules}{\overline{\Rules}}
\newcommand{\bx}{\overline{x}}
\newcommand{\ar}{\overline{r}}

% abstract counter systems
\newcommand{\acs}{\mathsf{ACS}}
\newcommand{\acsa}{\mathsf{ACS(\absA)}}
\newcommand{\avg}{\overline{\vg}}
\newcommand{\absConfs}{\overline{\Confs}}
\newcommand{\absconf}{\overline{\conf}}
\newcommand{\absTrans}{\overline{\Trans}}
\newcommand{\abcs}{(\absConfs, \absConfs_0, \absTrans)}
\newcommand{\bvarphi}{\overline{\varphi}}
\newcommand{\bphi}{\overline{\varphi}}

% 01-counter systems
\newcommand{\zcs}{\mathsf{ZCS}}
\newcommand{\zcsa}{\mathsf{ZCS(\absA)}}
\newcommand{\zConfs}{\Confs^z}
\newcommand{\zconf}{{\conf^{z}}}
\newcommand{\zpi}{\mathsf{\pi^z}}
\newcommand{\vkz}{\vec{k}^{z}}

% not checked
\newcommand{\vx}{\vec{x}}
\newcommand{\vc}{\vec{c}}
\newcommand{\vv}{\vec{v}}
\newcommand{\vs}{\vec{s}}
\newcommand{\vi}{\vec{i}}
\newcommand{\vcp}{\vec{c\scriptstyle'}}
\newcommand{\vcpp}{\vec{c\scriptstyle''}}
\newcommand{\vu}{\vec{u}}
\newcommand{\vxa}[1][]{\vec{x}_{A_{#1}}}
\newcommand{\vxb}[1][]{\vec{x}_{B_{#1}}}
\newcommand{\vcb}{\vec{c}^b}
\newcommand{\AP}{{\sf AP}}

\newcommand{\bigdotcup}{\mathaccent\cdot{\bigcup}}
\newcommand{\pclass}{{\sf cl}}
\newcommand{\val}{{\sf val}}
\newcommand{\guard}{{\sf guard}}
\newcommand{\eval}{{\sf eval}}
\newcommand{\init}{{\sf init}\xspace}
\newcommand{\templates}{T}
\newcommand{\templateI}{A}
\newcommand{\templateII}{B}
\newcommand{\formula}{F}
\newcommand{\formulas}{\mathcal{\formula}}
\newcommand{\inputs}{\Sigma}
\newcommand{\localin}{\sigma}
\newcommand{\globIn}{E}
\newcommand{\bpi}{\mathsf{\bar{\pi}}}


\definecolor{darkgreen}{rgb}{0,0.5,0}
\definecolor{darkblue}{rgb}{0,0,.5}
\definecolor{mygray}{gray}{.3}

%% macros for LTSs (states, state sets, etc.)
%% to change whole batch, just change \state command
\newcommand{\state}{q}
\newcommand{\states}{\textsf{set}}
\newcommand{\cstate}{s}
\newcommand{\cstateset}{S}
\newcommand{\initstate}{{\sf init}\xspace}
\newcommand{\stateset}{\expandafter\MakeUppercase\expandafter{\state}}
%\newcommand{\State}{s}
\newcommand{\Stateset}{\expandafter\MakeUppercase\expandafter{\State}}
\newcommand{\LTS}{\mathcal \stateset}
%% transition function, state labels, ...
\newcommand{\labeling}{L}
\newcommand{\labelings}{\mathcal{\labeling}}
\newcommand{\outputs}{O}
\newcommand{\time}{m}
\newcommand{\dead}{{\sf dead}\xspace}
\newcommand{\Enable}{{\sf Enable}\xspace}
\newcommand{\appears}{{\sf appears}\xspace}
\newcommand{\last}{{\sf last}\xspace}
\newcommand{\occurs}{{\sf occurs}\xspace}
\newcommand{\maxD}{{\sf maxD}\xspace}

\newcommand{\myloop}[1]{#1 \rightarrow \initstate \rightarrow #1\xspace}


%% set cardinality
\newcommand{\card}[1]{\left| {#1} \right|}
\newcommand{\Nat}{\mathbb{N}}
\newcommand{\Int}{\mathbb{Z}}
\newcommand{\Bool}{\mathbb{B}}
\newcommand{\phil}{\varphi^{\leq}}
\newcommand{\phiu}{\varphi^{>}}
\newcommand{\bphil}{\bar{\varphi}^{\leq}}
\newcommand{\bphiu}{\bar{\varphi}^{>}}
%% 'such that'
\newcommand{\|}{\mid}
\newcommand{\smi}{\!\setminus\!}
%% disjoint union
\newcommand{\cupdot}{\mathbin{\dot{\cup}}}

%paragraph with some skip and bold heading
\newcommand{\smartpar}[1]{\medskip \noindent {\bf #1.}}


%% specification, implication
\newcommand{\spec}{\Phi}
\newcommand{\pspec}{\Phi}
\newcommand{\impl}{\rightarrow}
\newcommand{\Impl}{\Rightarrow}
\newcommand{\Implied}{\Leftarrow}
\newcommand{\iff}{\Leftrightarrow}

%\newtheorem{thm}{Theorem}
%\newtheorem{lem}{Lemma}
%\newtheorem{cor}[thm]{Corollary}
%\newtheorem{exa}{Example}
%\newtheorem{obs}{Observation}
\newcommand{\disj}[1]{\exists\{#1\}}
\newcommand{\conj}[1]{\forall\{#1\}}

% Disj proof
\newcommand{\sys}[1]{A {\parallel} B^(#1)\xspace}%{(A,B)^{(#1)}}
\newcommand{\cutoffsys}{A {\parallel} B^c\xspace}%{\ensuremath{(A,B)^{(1,c)}}\xspace}
\newcommand{\largesys}{A {\parallel} B^n\xspace}%{\ensuremath{(A,B)^{(1,n)}}\xspace}
\newcommand{\largesyse}[1]{A {\parallel} B^{#1}\xspace}
\newcommand{\interleave}{\textit{interleave}\xspace}
%\newcommand{\first}{\textit{first}}
%\newcommand{\last}{\mathsf{last}}
\newcommand{\l}{\mathsf{l}}
\newcommand{\f}{\mathsf{f}}
\newcommand{\rpt}{\mathsf{rpt}}
% \newcommand{\ppi}{\mathsf{pi}}
\newcommand{\ppi}[1]{\mathsf{p}_{\circlearrowleft #1}}
\newcommand{\ei}[1]{\mathsf{e}_{\circlearrowleft #1}}
\newcommand{\qstar}{{q^\star}}
\newcommand{\estar}{{\mathsf{e}^\star}}
\newcommand{\fstar}{{\mathsf{f}^\star}}
\newcommand{\fin}{\textit{fin}}
\newcommand{\pref}{\mathsf{pref}}
\newcommand{\reach}{\mathsf{reach}}
\newcommand{\tpref}{\textit{pref}}

\newcommand{\inf}{\textit{inf}}
\newcommand{\lo}{\circlearrowleft}

\newcommand{\others}{\textit{others}}
\newcommand{\looop}{\textit{loop}}
\newcommand{\xloop}{\mathsf{xloop}}
\newcommand{\yloop}{\mathsf{yloop}}
\newcommand{\yround}{\mathsf{round}}
\newcommand{\inject}{\mathsf{inject}}

\newcommand{\Xrightarrow}[2]{\xrightarrow{#1}{}\negthickspace^{#2}\ }
\newcommand{\transition}[3]{#1 \xrightarrow{#2} #3}
\newcommand{\transitionP}[4]{#1 \Xrightarrow{#2}{#3} #4}
\newcommand{\slice}[2]{[{#1}\!:\!{#2}]}
\newcommand{\MinComp}[1]{\ensuremath{x(B^{\witfirst_{#1}})\slice{1}{\first_{#1}\!\!-\!\!1}}}
\newcommand{\flood}{\textit{flood}}
\newcommand{\yflood}{y^\flood}
\newcommand{\sflood}{s^\flood}
\newcommand{\eflood}{e^\flood}
\newcommand{\Pflood}{P^\flood}
\newcommand{\fair}{\textit{fair}}
\newcommand{\yfair}{y^\fair}
\newcommand{\sfair}{s^\fair}
\newcommand{\efair}{e^\fair}
\newcommand{\Pfair}{P^\fair}
\newcommand{\evac}{\textit{evac}}
\newcommand{\yevac}{y^\evac}
\newcommand{\sevac}{s^\evac}
\newcommand{\eevac}{e^\evac}
\newcommand{\Pevac}{P^\evac}
\newcommand{\qloop}{q}
\newcommand{\eloop}{e}

% add line breaks to table cells, to labels of tikz 
% usage: \specialcell{first line \\ second line} 
\newcommand{\specialcellC}[2][c]{%
  \begin{tabular}[#1]{@{}c@{}}#2\end{tabular}}
\newcommand{\specialcellL}[2][c]{%
  \begin{tabular}[#1]{@{}l@{}}#2\end{tabular}}


\newcommand{\CURRENT}{\hl{CURRENT}}

%% Latex `sugar' extensions
\newcommand{\li}{\begin{itemize}}
\newcommand{\il}{\end{itemize}}
\newcommand{\-}{\item}

\newcommand{\myparagraphraw}[1]{\smallskip\noindent{\bf#1}}
\newcommand{\myparagraph}[1]{\myparagraphraw{#1.}}

\newcommand{\ord}{\lesssim}
\newcommand{\zord}{\ord^z}
\usepackage{algorithm}
\usepackage[noend]{algpseudocode}
\usepackage{float}

\newcommand{\nuparrow}{\hspace{0.1cm}\uparrow \hspace{-0.1cm}}
\newcommand{\guardset}{\mathcal{G}}

\newcommand{\cbasis}{\begin{array}{l}
		CBasis  = \{ (\state_A,\vc) \in \cstateset \mid \exists (\state'_A,\vcp) \in R: \\
	\ \ \boldsymbol{[ \:} (\state_A,g,\state'_A) \in \delta_A \land (\state_A,\vc) \modelsg{\state_A} g 
	\land (\: (
	\vc=\vcp) \lor  ( \vcp(t) = 0 \land \vc=\vcp+\vu_t) \:) \boldsymbol{\:]}\\
	\ \ \lor \boldsymbol{[\:} (\state_i,\{q_t\},\state_j) \in \delta_B \land (\state_A,\vc) \modelsg{\state_i} g \land \state_A = \state'_A \\	
	\ \ \ \ \land \boldsymbol{(}\ \ \ \ ( \vc = \vcp + \vu_i - \vu_j ) \lor ( \vcp(t) = 0 \land \vcp(j) \geq 1 \land \vc = \vcp + \vu_i - \vu_j + \vu_t)\\
	\ \ \ \ \ \ \ \ \lor ( f \land \vc = \vcp + \vu_i ) \lor ( \: \vcp(t) = 0 \land \vcp(j)=0 \land \vc = \vc + \vu_i + \vu_t )\boldsymbol{\:\: )} \boldsymbol{\:]} \: \:\}.  ~~~~\qed
	\end{array}}
	
\newcommand{\globally}{\square}
\newcommand{\algorithmicthen}{}



%%%%%%%%%%%%%%%%%%%%%%%%%%%%%%%%%%%%%%%%%%%%%%%%%%%%%%%%%%%%%%%%%%%%%%%%%%%%%%%%
%%% LINE NUMBERS
%%%%%%%%%%%%%%%%%%%%%%%%%%%%%%%%%%%%%%%%%%%%%%%%%%%%%%%%%%%%%%%%%%%%%%%%%%%%%%%%
\usepackage[pagewise]{lineno} % switch, modulo
\newcommand\linenumberfont{\normalfont\tiny\sffamily\color{gray}}



%%%%%%%%%%%%%%%%%%%%%%%%%%%%%%%%%%%%%%%%%%%%%%%%%%%%%%%%%%%%%%%%%%%%%%%%%%%%%%%%
% TA Syntax
%%%%%%%%%%%%%%%%%%%%%%%%%%%%%%%%%%%%%%%%%%%%%%%%%%%%%%%%%%%%%%%%%%%%%%%%%%%%%%%%
\definecolor{bluekeywords}{HTML}{0052CC}
\definecolor{lightbluekeywords}{HTML}{009CC4}
\definecolor{greentypes}{HTML}{00CC44}
\definecolor{graynumbers}{rgb}{0.5, 0.5, 0.5}
\definecolor{greencomments}{rgb}{0.6, 0.4, 0.9}

\definecolor{orange}{RGB}{209, 106, 10}
\definecolor{violet}{RGB}{98, 4, 145}
\definecolor{red}{RGB}{148, 4, 4}


\usepackage[]{listings}

% Marker for general trasition systems with named states
\definecolor{cispa-blue}{HTML}{009CC4}
\definecolor{cispa-blue-20}{HTML}{D5ECF6}
\definecolor{cispa-anthracite}{HTML}{1C1C1C}

\definecolor{Gray}{gray}{0.925}


\section{Threshold Automata}\label{sec:TA}

This section introduces the underlying theoretical framework of TACO, which are (extended) threshold automata (TA)~\citep{KonnovVW14,KonnovVW17,BaumeisterEJSV24}.


%\begin{definition}\label{def:TA}
\textbf{Definition:}

\label{def:TA} A \emph{threshold automaton} (TA) \citep{KonnovVW14,KonnovVW17} is a tuple $A=(L, \mI, \Shared, \Pi,\Rules, RC)$ where:
\begin{itemize}[noitemsep,topsep=5pt]
\item $L$ is a finite set of \emph{locations}.%~(or local states).
%  We sometimes refer to locations by their indices in $L$.
%	
	\item $ \mI \subseteq L$ is the set of \emph{initial locations}.
%	
	\item $\Shared = \{x_0,\ldots,x_m\}$ is a finite set of \emph{shared variables} over $\Nat_0$.
%	
	\item $\Pi$ is a finite set of \emph{parameter variables} over $\Nat_0$.
	  %	
	\item $RC$, the \emph{resilience condition}, is a linear integer arithmetic formula over parameter variables.

	
	\item $\Rules$ is a set of rules, where a \emph{rule} is a tuple $r=(\from, \to, \varphi,\vup,\resets)$ such that:
	\begin{itemize}[noitemsep,topsep=5pt]
		\item $\from,\to \in L$.
%		
		\item $\varphi$ is a conjunction of lower guards and upper guards.
		A \emph{lower guard} is a constraint $a_0 + \sum_{i=1}^{|\Pi|} a_i \cdot p_i \leq x$, with $x \in \Shared$, $a_0,\ldots,a_{|\Pi|} \in \mathbb{Q}$, $p_1,\ldots,p_{|\Pi|} \in \Pi$. An \emph{upper guard} is a constraint $ a_0 + \sum_{i=1}^{|\Pi|} a_i \cdot p_i > x$.\\		
The left-hand side of a lower or upper guard is called a \emph{threshold}. 
%		
		\item $\vup \in \Int^{|\Shared|}$ is an \emph{update vector} for shared variables.
%	
	\item $\resets \subseteq \Shared$ is the set of shared variables to be reset to 0.
	\end{itemize}
%
\end{itemize}

%\end{definition}

Most commonly, the parameter variables are $\Pi = \{n,t,f\}$, where $n$ is the number of processes, $t$ is a bound on the number of faulty processes, and $f$ is the actual number of faulty processes.
This allows $RC$ to express assumptions about the fraction of faulty processes in the system, e.g., $RC = n > 3t\, \land\, t \geq f$.

If $\vup \in \Nat_0^{|\Shared|}$ and no variables are reset, then the TA is called a \emph{canonical TA} (CTA). 
Otherwise we call it an \emph{extended TA} (ETA)\citep{BaumeisterEJSV24}. 

Note that rechability is in general undecidable for ETA~\citep{Kukovec0W18,BaumeisterEJSV24}. 
Consequently, TACO's decision procedures become semi-decision procedures for the ETAs, which will in general not terminate when an ETA is safe.


\subsection{Semantics}

For a vector $\vp \in \Nat_0^{|\Pi|}$, we write $\vp \models RC$ if $RC$ holds after substituting parameter variables with values according to $\vp$.
Then the set of \emph{admissible} parameters is $\vP_{RC} = \{\vp \in \Nat_0^{|\Pi|} : \vp \models RC\}$.

Once a TA is specified, the number of processes to be modeled must also be defined through the function $\mN: \vP_{RC} \rightarrow \Nat_0$ (usually $\mN(n,t,f) = n - f$).


\textbf{Configurations.} A \emph{configuration} of a TA is a triple $\sigma = (\vk, \vg, \vp)$, where $\vk \in \Nat_0^{|L|}$ assigns a number of processes to each location, $\vg \in \Nat_0^{|\Shared|}$ assigns a valuation to the shared variables, and $\vp \in \vP_{RC}$ assigns a valuation to the parameters.
We say that a configuration is \emph{initial} if $\vg = \vec{0}$ and $\forall i$ $\vk[i] > 0 \iff l_i \in \mI$.


\textbf{Paths.} A rule $r = (\from, \to, \varphi,\vup,\resets)$ is \emph{enabled} in a configuration $\sigma = (\vk,\vg,\vp)$ if $\vec{k}[\from] > 0$ and $(\vg,\vp)$ satisfies the guard $\varphi$.
In this case, we say that $r(\sigma) = (\vk',\vg',\vp')$ where $\vk'[\from]=\vk[\from] -1$, $\vk'[\to]=\vk[\to] + 1$, $\vg'=\vg + \vup$, and $\vp'=\vp$.
A \emph{path} $\pi$ is then a finite or infinite sequence of configurations $\pi = \sigma_0, \sigma_1, \ldots$ such that for each $i$ there exists an enabled rule $r_i$ with $\sigma_{i+1} = r_i(\sigma_i)$.
We denote by $\Paths$ the set of all paths $\pi$ in the threshold automaton.

\vspace{-2em}
\begin{figure}[h]
	\centering
	\includegraphics[width=0.8\textwidth]{resources/flood-min-pseudo}
	\caption{Pseudocode of a reliable broadcast protocol.}\label{fig:alg-floodmin}
  \end{figure}
\vspace{-1.5em}


\textbf{Example:} 

Figure~\ref{fig:alg-floodmin} shows the pseudocode of a reliable broadcast, inspired by~\citep{srikanth1987simulating}.
In every round, processes with input $v=1$ will broadcast a message, then processes with $v=0$ will set $v=1$ if they received at least $t+1$ messages, and finally processes set $accept=true$ if at least $n-t$ messages have been received. If $accept$ is false at the end of the round, a new round starts. 

\vspace{-2em}
\begin{figure}[h]
	\centering
	\includegraphics[width=0.7\textwidth]{resources/flood-min-ta}
	\caption{Threshold automaton of the algorithm in Figure~\ref{fig:alg-floodmin}.}\label{fig:ta-floodmin}
  \end{figure}
\vspace{-1.5em}

\textbf{Example:} Figure~\ref{fig:ta-floodmin} depicts a TA for this algorithm, with 
$L = \{V_0, V_1, RV_0, SE, AC\}$, $\mI = \{V_0, V_1\}$, $\Shared = \{nsnt, rec\}$, $\Pi = \{n, t, f\}$, and 
$RC = n > 3t \land t \geq f \geq 0$.
A process in $V_1$ has input $1$ and can move freely (there is no guard) to $SE$, incrementing both variables $nsnt$ and $rec$ to simulate lines~4--6 of the algorithm. 
A process in location $V_0$ has input $0$ and can move to $RV_0$, incrementing $rec$ to simulate line~6 (but not line 5, since the condition in line~4 evaluates to false). 
From $RV_0$, a process can move to $SE$ if $nsnt \geq \varphi_1$, simulating lines~$7$-$8$.
Processes that started with $v=1$ already are in $SE$, corresponding to the fact that $v=1$ is not changed in line 8.\\

Note that instead of being $t+1$, $\varphi_1$ is chosen as $t + 1 - f$.
This prevents processes from making a transition based on messages from faulty processes, hence the TA represents the behavior of correct processes and $f$ the effect of faulty processes on correct ones.
For more details on the role of $f$, see~\citep{Stoilkovska0WZ20}. \\

Processes in $SE$ can move to $AC$ if $nsnt \geq \varphi_2$, simulating lines~$9$-$10$.
If $nsnt < \varphi_2$ and $rec \geq \phi_3$ (this constraint corresponds to waiting long enough for all non-faulty processes to receive all messages), the condition in line~9 was not satisfied and a new iteration of the while-loop is started by moving back from $SE$ to $V_1$.
The first process taking this transition will reset $nsnt$ and $rec$, the others take the transition that does not change any variables.
Similarly, processes in $RV_0$ will move to $V_0$ to start the new round.\\

The $RC$ condition is critical here to ensure fault-tolerance: if $f > t$, then $AC$ will be reachable even if all processes start in $V_0$, violating the validity property.
