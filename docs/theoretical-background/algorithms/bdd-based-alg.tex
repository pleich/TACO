\subsection{ZCS Model Checking Algorithm}\label{alg:bdd}

\begin{figure}[h]
    \centering
    \includegraphics[width=\textwidth]{./resources/zcs-algorithm.webp}
    \caption{Workflow of the ZCS model checking algorithm.}
\end{figure}

The algorithm follows a workflow similar to the ACS-based approach in \cref{sec:alg-wsts} but operates over a different abstract model, the \emph{$01$-counter system} (ZCS).
A \emph{ZCS} can be seen as a further abstraction of the abstract counter system, where integer counters are replaced by binary counters.
That is, if the counter of location $i$ is set to 1, this indicates that one or more processes reside in $i$.
This abstraction has two important consequences.
First, the ZCS is a finite-state model, as each counter takes only two values.
Second, the finite state space can be represented compactly and manipulated efficiently using Binary Decision Diagrams (BDDs) \citep{bdds}.
Therefore, given a TA and a reachability specification, the ZCS algorithm first constructs the corresponding ATA, then creates the ZCS and the set of ZCS error configurations defined by the specification.
Since a ZCS is finite-state, we perform a BDD-based backward exploration from the error configurations until reaching a fixed point.
If no initial configuration is included in the fixed point, the TA is deemed correct.
If an initial configuration is reached, an \emph{error graph} is generated that encodes all potential paths from an initial configuration to an error configuration.

As in the ACS-based approach, the abstraction may introduce spurious error paths.
To eliminate them, each abstract path in the error graph is checked using an SMT solver.
If a concrete error path is discovered, the TA is declared incorrect.
Otherwise all abstract paths are spurious and the TA is correct.
For ETAs, the procedure is only a semi-decision procedure.

\textbf{Implementation Details.}
As an input, this algorithm uses a set of IntervalTA (see \cref{fig:architecture}), as explained in the implementation details of~\cref{sec:alg-wsts}.

As a heuristic, a variant of this algorithm can also be used to check only whether the error graph is empty. 
This requires less bookkeeping than in the standard algorithm and allows it to terminate even faster (at the cost of only giving a definitive answer about correctness if the error graph really is empty).

If an error graph is constructed, then spuriosity checks are done breadth-first, i.e., starting with the shortest paths from initial states to error states.

\textbf{Usage.} To use the ZCS model checking algorithm with TACO, the user needs to specify the option \texttt{zcs} when invoking the tool from the command line.
Additionally, the user can choose a heuristic for exploring the error graph depending on the input TA by specifying the option \texttt{--heuristic}.
Since TACO automatically detects whether the input TA is a MTA or an ETA, this flag does not need to be set by the user to choose a suitable heuristic.
However, if the user wants to check only whether the error graph is empty, they can set the flag to \texttt{empty-error-graph-heuristic}.
