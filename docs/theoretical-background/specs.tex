\usepackage{subcaption}
\usepackage{xcolor, colortbl}
\usepackage{amsfonts}
\usepackage{amsmath,thm-restate}
\usepackage{amssymb}
\usepackage{temporal}
\usepackage{cite}
\usepackage{graphicx}
\usepackage[inline]{enumitem}
\usepackage{wrapfig}
\usepackage{environ}
\usepackage{multirow}
\usepackage{xspace}
\usepackage{color}
\usepackage{thmtools}
\usepackage{makecell}
\usepackage{pgf}
\usepackage{caption}
\usepackage{mdframed}
\definecolor{mygreen}{RGB}{0,150,0}
\usepackage{pifont} % symboles
\newcommand{\xmark}{\leavevmode{\color{red}\ding{55}}}%
\newcommand{\cmark}{\leavevmode{\color{green!75!black}\ding{51}}}%


\newcommand{\mailto}[1]{\href{mailto:#1}{\nolinkurl{#1}}}

% general logic notation
\newcommand{\modelsg}[1]{\models_{#1}}
\newcommand{\mT}{\mathcal{T}}
\newcommand{\mTz}{\mathcal{T}^{z}}
\newcommand{\mR}{\mathcal{R}}
\newcommand{\mI}{\mathcal{I}}
\newcommand{\mC}{\mathcal{C}}
\newcommand{\mD}{\mathcal{D}}
\newcommand{\mF}{\mathcal{F}}
\newcommand{\mX}{\mathcal{X}}
\newcommand{\mM}{\mathcal{M}}
\newcommand{\mN}{\mathcal{N}}
\newcommand{\mA}{\mathcal{A}}
\newcommand{\mP}{\mathcal{P}}
\newcommand{\mB}{\mathcal{B}}
\newcommand{\mU}{\mathcal{U}}
\newcommand{\mS}{\mathcal{S}}
\newcommand{\mE}{\mathcal{E}}
\newcommand{\mQ}{\mathcal{Q}}
\newcommand{\mAR}{\mathcal{AR}}
\newcommand{\mTH}{\mathcal{TH}}
\newcommand{\bbN}{\mathbb{N}}
\newcommand{\bbB}{\mathbb{B}}
\newcommand{\bbR}{\mathbb{R}}


% general logic notation
\newcommand{\vec}[1]{\mathbf{#1}}

% (non-abstract) threshold automata
\newcommand{\resets}{\tau}
\newcommand{\Shared}{\Gamma}
\newcommand{\vP}{\vec{P}}
\newcommand{\vp}{\vec{p}}
\newcommand{\from}{\textit{from}}
\newcommand{\to}{\textit{to}}
\newcommand{\vup}{\vec{uv}}
\newcommand{\Rules}{\mR}
\newcommand{\Paths}{\mP(TA)}


% (non-abstract) counter systems
\newcommand{\cs}{\mathsf{CS}}
\newcommand{\csa}{\mathsf{CS(A)}}
\newcommand{\vg}{\vec{g}}
\newcommand{\vk}{\vec{k}}
\newcommand{\Confs}{\Sigma}
\newcommand{\conf}{\sigma}
\newcommand{\Trans}{\mT}

% abstract threshold automata
\newcommand{\absTA}{\overline{\text{TA}}}
\newcommand{\absA}{\overline{A}}
\newcommand{\absShared}{\overline{\Shared}}
\newcommand{\absRules}{\overline{\Rules}}
\newcommand{\bx}{\overline{x}}
\newcommand{\ar}{\overline{r}}

% abstract counter systems
\newcommand{\acs}{\mathsf{ACS}}
\newcommand{\acsa}{\mathsf{ACS(\absA)}}
\newcommand{\avg}{\overline{\vg}}
\newcommand{\absConfs}{\overline{\Confs}}
\newcommand{\absconf}{\overline{\conf}}
\newcommand{\absTrans}{\overline{\Trans}}
\newcommand{\abcs}{(\absConfs, \absConfs_0, \absTrans)}
\newcommand{\bvarphi}{\overline{\varphi}}
\newcommand{\bphi}{\overline{\varphi}}

% 01-counter systems
\newcommand{\zcs}{\mathsf{ZCS}}
\newcommand{\zcsa}{\mathsf{ZCS(\absA)}}
\newcommand{\zConfs}{\Confs^z}
\newcommand{\zconf}{{\conf^{z}}}
\newcommand{\zpi}{\mathsf{\pi^z}}
\newcommand{\vkz}{\vec{k}^{z}}

% not checked
\newcommand{\vx}{\vec{x}}
\newcommand{\vc}{\vec{c}}
\newcommand{\vv}{\vec{v}}
\newcommand{\vs}{\vec{s}}
\newcommand{\vi}{\vec{i}}
\newcommand{\vcp}{\vec{c\scriptstyle'}}
\newcommand{\vcpp}{\vec{c\scriptstyle''}}
\newcommand{\vu}{\vec{u}}
\newcommand{\vxa}[1][]{\vec{x}_{A_{#1}}}
\newcommand{\vxb}[1][]{\vec{x}_{B_{#1}}}
\newcommand{\vcb}{\vec{c}^b}
\newcommand{\AP}{{\sf AP}}

\newcommand{\bigdotcup}{\mathaccent\cdot{\bigcup}}
\newcommand{\pclass}{{\sf cl}}
\newcommand{\val}{{\sf val}}
\newcommand{\guard}{{\sf guard}}
\newcommand{\eval}{{\sf eval}}
\newcommand{\init}{{\sf init}\xspace}
\newcommand{\templates}{T}
\newcommand{\templateI}{A}
\newcommand{\templateII}{B}
\newcommand{\formula}{F}
\newcommand{\formulas}{\mathcal{\formula}}
\newcommand{\inputs}{\Sigma}
\newcommand{\localin}{\sigma}
\newcommand{\globIn}{E}
\newcommand{\bpi}{\mathsf{\bar{\pi}}}


\definecolor{darkgreen}{rgb}{0,0.5,0}
\definecolor{darkblue}{rgb}{0,0,.5}
\definecolor{mygray}{gray}{.3}

%% macros for LTSs (states, state sets, etc.)
%% to change whole batch, just change \state command
\newcommand{\state}{q}
\newcommand{\states}{\textsf{set}}
\newcommand{\cstate}{s}
\newcommand{\cstateset}{S}
\newcommand{\initstate}{{\sf init}\xspace}
\newcommand{\stateset}{\expandafter\MakeUppercase\expandafter{\state}}
%\newcommand{\State}{s}
\newcommand{\Stateset}{\expandafter\MakeUppercase\expandafter{\State}}
\newcommand{\LTS}{\mathcal \stateset}
%% transition function, state labels, ...
\newcommand{\labeling}{L}
\newcommand{\labelings}{\mathcal{\labeling}}
\newcommand{\outputs}{O}
\newcommand{\time}{m}
\newcommand{\dead}{{\sf dead}\xspace}
\newcommand{\Enable}{{\sf Enable}\xspace}
\newcommand{\appears}{{\sf appears}\xspace}
\newcommand{\last}{{\sf last}\xspace}
\newcommand{\occurs}{{\sf occurs}\xspace}
\newcommand{\maxD}{{\sf maxD}\xspace}

\newcommand{\myloop}[1]{#1 \rightarrow \initstate \rightarrow #1\xspace}


%% set cardinality
\newcommand{\card}[1]{\left| {#1} \right|}
\newcommand{\Nat}{\mathbb{N}}
\newcommand{\Int}{\mathbb{Z}}
\newcommand{\Bool}{\mathbb{B}}
\newcommand{\phil}{\varphi^{\leq}}
\newcommand{\phiu}{\varphi^{>}}
\newcommand{\bphil}{\bar{\varphi}^{\leq}}
\newcommand{\bphiu}{\bar{\varphi}^{>}}
%% 'such that'
\newcommand{\|}{\mid}
\newcommand{\smi}{\!\setminus\!}
%% disjoint union
\newcommand{\cupdot}{\mathbin{\dot{\cup}}}

%paragraph with some skip and bold heading
\newcommand{\smartpar}[1]{\medskip \noindent {\bf #1.}}


%% specification, implication
\newcommand{\spec}{\Phi}
\newcommand{\pspec}{\Phi}
\newcommand{\impl}{\rightarrow}
\newcommand{\Impl}{\Rightarrow}
\newcommand{\Implied}{\Leftarrow}
\newcommand{\iff}{\Leftrightarrow}

%\newtheorem{thm}{Theorem}
%\newtheorem{lem}{Lemma}
%\newtheorem{cor}[thm]{Corollary}
%\newtheorem{exa}{Example}
%\newtheorem{obs}{Observation}
\newcommand{\disj}[1]{\exists\{#1\}}
\newcommand{\conj}[1]{\forall\{#1\}}

% Disj proof
\newcommand{\sys}[1]{A {\parallel} B^(#1)\xspace}%{(A,B)^{(#1)}}
\newcommand{\cutoffsys}{A {\parallel} B^c\xspace}%{\ensuremath{(A,B)^{(1,c)}}\xspace}
\newcommand{\largesys}{A {\parallel} B^n\xspace}%{\ensuremath{(A,B)^{(1,n)}}\xspace}
\newcommand{\largesyse}[1]{A {\parallel} B^{#1}\xspace}
\newcommand{\interleave}{\textit{interleave}\xspace}
%\newcommand{\first}{\textit{first}}
%\newcommand{\last}{\mathsf{last}}
\newcommand{\l}{\mathsf{l}}
\newcommand{\f}{\mathsf{f}}
\newcommand{\rpt}{\mathsf{rpt}}
% \newcommand{\ppi}{\mathsf{pi}}
\newcommand{\ppi}[1]{\mathsf{p}_{\circlearrowleft #1}}
\newcommand{\ei}[1]{\mathsf{e}_{\circlearrowleft #1}}
\newcommand{\qstar}{{q^\star}}
\newcommand{\estar}{{\mathsf{e}^\star}}
\newcommand{\fstar}{{\mathsf{f}^\star}}
\newcommand{\fin}{\textit{fin}}
\newcommand{\pref}{\mathsf{pref}}
\newcommand{\reach}{\mathsf{reach}}
\newcommand{\tpref}{\textit{pref}}

\newcommand{\inf}{\textit{inf}}
\newcommand{\lo}{\circlearrowleft}

\newcommand{\others}{\textit{others}}
\newcommand{\looop}{\textit{loop}}
\newcommand{\xloop}{\mathsf{xloop}}
\newcommand{\yloop}{\mathsf{yloop}}
\newcommand{\yround}{\mathsf{round}}
\newcommand{\inject}{\mathsf{inject}}

\newcommand{\Xrightarrow}[2]{\xrightarrow{#1}{}\negthickspace^{#2}\ }
\newcommand{\transition}[3]{#1 \xrightarrow{#2} #3}
\newcommand{\transitionP}[4]{#1 \Xrightarrow{#2}{#3} #4}
\newcommand{\slice}[2]{[{#1}\!:\!{#2}]}
\newcommand{\MinComp}[1]{\ensuremath{x(B^{\witfirst_{#1}})\slice{1}{\first_{#1}\!\!-\!\!1}}}
\newcommand{\flood}{\textit{flood}}
\newcommand{\yflood}{y^\flood}
\newcommand{\sflood}{s^\flood}
\newcommand{\eflood}{e^\flood}
\newcommand{\Pflood}{P^\flood}
\newcommand{\fair}{\textit{fair}}
\newcommand{\yfair}{y^\fair}
\newcommand{\sfair}{s^\fair}
\newcommand{\efair}{e^\fair}
\newcommand{\Pfair}{P^\fair}
\newcommand{\evac}{\textit{evac}}
\newcommand{\yevac}{y^\evac}
\newcommand{\sevac}{s^\evac}
\newcommand{\eevac}{e^\evac}
\newcommand{\Pevac}{P^\evac}
\newcommand{\qloop}{q}
\newcommand{\eloop}{e}

% add line breaks to table cells, to labels of tikz 
% usage: \specialcell{first line \\ second line} 
\newcommand{\specialcellC}[2][c]{%
  \begin{tabular}[#1]{@{}c@{}}#2\end{tabular}}
\newcommand{\specialcellL}[2][c]{%
  \begin{tabular}[#1]{@{}l@{}}#2\end{tabular}}


\newcommand{\CURRENT}{\hl{CURRENT}}

%% Latex `sugar' extensions
\newcommand{\li}{\begin{itemize}}
\newcommand{\il}{\end{itemize}}
\newcommand{\-}{\item}

\newcommand{\myparagraphraw}[1]{\smallskip\noindent{\bf#1}}
\newcommand{\myparagraph}[1]{\myparagraphraw{#1.}}

\newcommand{\ord}{\lesssim}
\newcommand{\zord}{\ord^z}
\usepackage{algorithm}
\usepackage[noend]{algpseudocode}
\usepackage{float}

\newcommand{\nuparrow}{\hspace{0.1cm}\uparrow \hspace{-0.1cm}}
\newcommand{\guardset}{\mathcal{G}}

\newcommand{\cbasis}{\begin{array}{l}
		CBasis  = \{ (\state_A,\vc) \in \cstateset \mid \exists (\state'_A,\vcp) \in R: \\
	\ \ \boldsymbol{[ \:} (\state_A,g,\state'_A) \in \delta_A \land (\state_A,\vc) \modelsg{\state_A} g 
	\land (\: (
	\vc=\vcp) \lor  ( \vcp(t) = 0 \land \vc=\vcp+\vu_t) \:) \boldsymbol{\:]}\\
	\ \ \lor \boldsymbol{[\:} (\state_i,\{q_t\},\state_j) \in \delta_B \land (\state_A,\vc) \modelsg{\state_i} g \land \state_A = \state'_A \\	
	\ \ \ \ \land \boldsymbol{(}\ \ \ \ ( \vc = \vcp + \vu_i - \vu_j ) \lor ( \vcp(t) = 0 \land \vcp(j) \geq 1 \land \vc = \vcp + \vu_i - \vu_j + \vu_t)\\
	\ \ \ \ \ \ \ \ \lor ( f \land \vc = \vcp + \vu_i ) \lor ( \: \vcp(t) = 0 \land \vcp(j)=0 \land \vc = \vc + \vu_i + \vu_t )\boldsymbol{\:\: )} \boldsymbol{\:]} \: \:\}.  ~~~~\qed
	\end{array}}
	
\newcommand{\globally}{\square}
\newcommand{\algorithmicthen}{}



%%%%%%%%%%%%%%%%%%%%%%%%%%%%%%%%%%%%%%%%%%%%%%%%%%%%%%%%%%%%%%%%%%%%%%%%%%%%%%%%
%%% LINE NUMBERS
%%%%%%%%%%%%%%%%%%%%%%%%%%%%%%%%%%%%%%%%%%%%%%%%%%%%%%%%%%%%%%%%%%%%%%%%%%%%%%%%
\usepackage[pagewise]{lineno} % switch, modulo
\newcommand\linenumberfont{\normalfont\tiny\sffamily\color{gray}}



%%%%%%%%%%%%%%%%%%%%%%%%%%%%%%%%%%%%%%%%%%%%%%%%%%%%%%%%%%%%%%%%%%%%%%%%%%%%%%%%
% TA Syntax
%%%%%%%%%%%%%%%%%%%%%%%%%%%%%%%%%%%%%%%%%%%%%%%%%%%%%%%%%%%%%%%%%%%%%%%%%%%%%%%%
\definecolor{bluekeywords}{HTML}{0052CC}
\definecolor{lightbluekeywords}{HTML}{009CC4}
\definecolor{greentypes}{HTML}{00CC44}
\definecolor{graynumbers}{rgb}{0.5, 0.5, 0.5}
\definecolor{greencomments}{rgb}{0.6, 0.4, 0.9}

\definecolor{orange}{RGB}{209, 106, 10}
\definecolor{violet}{RGB}{98, 4, 145}
\definecolor{red}{RGB}{148, 4, 4}


\usepackage[]{listings}

% Marker for general trasition systems with named states
\definecolor{cispa-blue}{HTML}{009CC4}
\definecolor{cispa-blue-20}{HTML}{D5ECF6}
\definecolor{cispa-anthracite}{HTML}{1C1C1C}

\definecolor{Gray}{gray}{0.925}


\section{Specifications on Threshold Automata}\label{sec:specs}


\subsection{Basic Specifications}

We consider two basic kinds of specifications: \emph{coverability}, which refers to the notion of coverability that is widely used in parametrized verification, e.g., Petri nets~\citep{Finkel91} or VASS~\citep{rackoff1978covering},  and \emph{reachability}.


\textbf{Definition:} The \emph{parametrized coverability problem} is: Given a TA and a \emph{coverability specification} $\vec{c} \in \Nat_0^{|L|}$, decide if there is a reachable configuration $\conf_r$\footnote{A configuration $\conf_r$ is said to be reachable if there exists a path starting from an initial configuration and ending in $\conf_r$.} with $\conf_r.\vk[i] \geq \vec{c}[i]$ for all $0 \leq i \leq |L|-1$.


\textbf{Definition:} The \emph{parametrized reachability problem} is: Given a TA and a \emph{reachability specification} $\vec{c_r} \in \Nat_0^{|L|}$, decide if there is a reachable configuration $\conf_r$ with $\conf_r.\vk[i] \geq \vec{c_r}[i] \land \left( \conf_r.\vk[i] = 0 \iff \vec{c_r}[i] = 0 \right)$ for all $0 \leq i \leq |L|-1$.

Usually, our specifications are definitions of error configurations, and therefore paths that reach them are called \emph{error paths}.


\subsection{$ELTL_{FT}$} 

TACO supports safety specifications in the fragment of $ELTL_{FT}$~\citep{KonnovLVW17,KW18} that can be translated into reachability or coverability properties. Formally, the syntax we support is defined as follows:

\begin{align*}
\psi &::= \mathsf{pform} \;\;|\;\; \globally \psi \;\;|\;\; \psi \wedge \psi \\
\mathsf{pform} &::= \mathsf{cform} \;\;|\;\; \mathsf{gform} \vee \mathsf{cform} \\
\mathsf{cform} &::= \bigwedge_{l \in Locs} \conf.\vk[l] \neq 0
       \;\;|\;\; \bigvee_{l \in Locs} \conf.\vk[l] = 0
       \;\;|\;\; \mathsf{cform} \wedge \mathsf{cform} \\
\mathsf{gform} &::= \varphi
       \;\;|\;\; \neg \mathsf{gform}
       \;\;|\;\; \mathsf{gform} \wedge \mathsf{gform}
\end{align*}
where $\varphi$ is a guard as defined in \cref{sec:TA} and $Locs \subseteq L$.

\noindent
Furthermore, TACO also supports lower integer bounds on the number of processes in a specific location. 
The extended $\mathsf{cform}$ syntax is:

\begin{align*}
\mathsf{cform} ::= \conf.\vk[l] * n 
        \;\;|\;\; \conf.\vk[l] = 0 
        \;\;|\;\; \conf.\vk[l] \neq 0 
        \;\;|\;\; \mathsf{cform} \wedge \mathsf{cform}
        \;\;|\;\; \mathsf{cform} \vee \mathsf{cform}
\end{align*}

\noindent
where $l \in L, n \in \mathbb{N}$ and $* \in \{>, \geq\}$.


\textbf{Example:} One important property for consensus-based distributed algorithms is to check for \emph{validity}, i.e., if no process received a specific value (e.g., 0), then no process should decide on that value.
\ref{fig:alg-floodmin} specifies the valdity property in $ELTL_{FT}$ as follows:
\[V1 == 0 \implies \globally (AC == 0)\]
To check if this property holds, TACO transforms this formula into a coverability specification that requires at least one process to be eventually in location $AC$ while all correct processes initially start in location $V1$.
If such a configuration is reachable, then the validity property is violated.
Otherwise the property holds.
